\chapter{Grundlagen}
\label{cha:Grundlagen}

Zielgerichtete theoretische Grundlagen, sowohl fachliche, wie auch methodische.

Zu den Grundlagen gehören z.~B. auch Details zur Problemstellung, Stand der Technik und weitere Grundlagen, welche zur Konzeptausarbeitung, Umsetzung und Verifikation erforderlich sind.

Grundlagen haben immer einen Bezug zu den nachfolgenden Kapiteln. Diesen Bezug sollte man gelegentlich explizit herstellen, damit bereits in diesem Kapitel klar ist, wo und für was die Grundlagen gebraucht und angewandt werden.

\subsection{RFID und NFC-Technologie}



\subsection{Das MQTT-Protokoll}

\subsection{Node-Red - Eine Steuerungs- und Visualisierungssoftware}

\subsection{Speicherprogrammierte Steuerungen (SPS)}