\chapter{Einleitung}
\label{cha:Einleitung}

Mit der zunehmenden Automatisierung industrieller Produktionsprozesse wächst der Bedarf an intelligenten Systemen, die eine lückenlose Nachverfolgbarkeit und effiziente Steuerung ermöglichen.
Besonders in vernetzten Fertigungsumgebungen, wie sie im Kontext von Industrie-4.0 realisiert werden, spielt die automatische Identifikation von Produkten entlang der Fertigungskette eine zentrale Rolle. Zum Einsatz kommen dafür verschiedene Technologien - von optischer Erkennung bis hin zu RFID und NFC (Near Field Communication).

Die Lernfabrik der DHBW bietet eine reale, teilautomatisierte Fertigungsumgebung, in der solche Konzepte unter praxisnahen Bedingungen erprobt werden können. Aktuell fehlt jedoch ein System zur durchgängigen Identifikation und Nachverfolgung von Produkten innerhalb der Fertigungslinie. Dies erschwert nicht nur die Prozessüberwachung, sondern limitiert auch Möglichkeiten zur Datenanalyse, Optimierung und Dokumentation.

Ziel dieser Arbeit ist die Entwicklung eines Systems, das die Position und Identität von Produkten während des Fertigungsprozesses mit Hilfe von NFC-Sensoren automatisch erfasst. Zusätzlich soll es möglich sein Informationen automatisch auf die NFC-Tag zu schreiben. Die erfassten Daten werden an ein bestehendes Manufacturing Execution System (MES) übermittelt und dort zur Visualisierung bereitgestellt. Darüber hinaus sollen die von den NFC-Sensoren erfassten Informationen genutzt werden, um den logischen Ablauf des Bandumlaufsystems zu steuern. Hierfür ist ein Kommunikationsmechanismus zwischen den einzelnen Bandumlaufstationen und den NFC-Sensoren zu entwerfen. Im Rahmen dieser Arbeit wird eine geeignete Systemarchitektur für die Hard- und Softwarekomponenten konzipiert, implementiert und evaluiert.

Die zentrale Fragestellung der Arbeit lautet:
\begin{quote}
	\textit{Wie kann ein robustes, kosteneffizientes und erweiterbares Trackingsystem für Produkte in einer bestehenden Fertigungsumgebung mithilfe von NFC-Technologie gestaltet und in ein MES integriert werden?}
\end{quote}


Zur Beantwortung dieser Frage werden klassische ingenieurwissenschaftliche Methoden wie Anforderungsanalyse, Systementwurf, prototypische Umsetzung und Validierung angewendet. Neben der technischen Umsetzung liegt ein besonderes Augenmerk auf der industriellen Tauglichkeit sowie der zukünftigen Erweiterbarkeit des Systems.

Die Arbeit gliedert sich wie folgt:  
Kapitel~2 beschreibt die theoretischen und technischen Grundlagen der verwendeten Technologien. Kapitel~3 analysiert die Anforderungen an das System und erläutert das methodische Vorgehen. Kapitel~4 enthält den Entwurf der Systemarchitektur sowie die Darstellung und Auswertung der Ergebnisse. Abschließend fasst Kapitel~5 die wesentlichen Erkenntnisse zusammen und gibt einen Ausblick auf mögliche Weiterentwicklungen.
