\chapter*{Kurzfassung} %*-Variante sorgt dafür, das Abstract nicht im Inhaltsverzeichnis auftaucht

Die Lernfabrik produziert teilautomatisiert über eine Fertigungskette verschiedene Produkte. Zur Verbesserung der Steuerung, Nachverfolgbarkeit und Dokumentation soll ein System entwickelt werden, das die Position der Produkte mittels NFC-Technologie erfasst und die entsprechenden Daten an ein Manufacturing Execution System (MES) übermittelt. Während geeignete NFC-Tags bereits identifiziert wurden, muss die restliche Systemarchitektur standortspezifisch konzipiert, implementiert und getestet werden.

Ziel dieser Arbeit ist die Entwicklung eines industrietauglichen Systems zur positionsbasierten Produktverfolgung innerhalb der Fertigung. Die erfassten Informationen sollen in Echtzeit an ein vorhandenes System übertragen und dort visualisiert werden.

Zur Entwicklung der Systemarchitektur kamen klassische ingenieurwissenschaftliche Methoden zum Einsatz. Der Fokus lag auf einem kosteneffizienten, störungstoleranten und modular erweiterbaren Aufbau der Hardware- und Softwarekomponenten.

Die Arbeit mündete in einem vollständigen Systementwurf zur produktionsbegleitenden Datenerfassung über NFC. Realisiert wurde ein Prototyp bestehend aus dezentralen Erfassungseinheiten und einer zentralen Steuereinheit, die eine Schnittstelle zum MES bereitstellt. Neben der Hardware wurden die erforderlichen Softwarekomponenten zur Datenübertragung und -visualisierung sowie eine technische Dokumentation für zukünftige Erweiterungen erstellt.

\clearpage

\chapter*{Abstract} %*-Variante sorgt dafür, das Abstract nicht im Inhaltsverzeichnis auftaucht

The learning factory produces various products semi-automatically via a production chain. To improve control, traceability, and documentation, a system is to be developed that records the position of the products using NFC technology and transmits the corresponding data to a Manufacturing Execution System (MES). While suitable NFC tags have already been identified, the remaining system architecture must be designed, implemented, and tested site-specifically.

The goal of this work is to develop an industrial-grade system for position-based product tracking within production. The recorded information is to be transmitted in real time to an existing system and visualized there.

Classical engineering methods were used to develop the system architecture. The focus was on a cost-effective, fault-tolerant, and modularly expandable design of the hardware and software components.

The work culminated in a complete system design for production-related data acquisition via NFC. A prototype was realized consisting of decentralized acquisition units and a central control unit that provides an interface to the MES. In addition to the hardware, the necessary software components for data transmission and visualization as well as technical documentation for future extensions were created.

\clearpage