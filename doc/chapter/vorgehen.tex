\chapter{Konzeptentwurf}
\label{cha:vorgehen}

Dieses Kapitel beschreibt die methodische Herleitung eines technischen Konzepts zur Umsetzung des in Kapitel~\ref{cha:Grundlagen} vorgestellten Systems. Basierend auf den identifizierten technologischen Grundlagen und Rahmenbedingungen erfolgt zunächst eine strukturierte Analyse der funktionalen und nicht-funktionalen Anforderungen mithilfe etablierter Methoden des Requirements Engineerings.

Darauf aufbauend werden verschiedene Lösungsoptionen erarbeitet, hinsichtlich ihrer technischen und wirtschaftlichen Eignung bewertet und schließlich zu einem konsistenten Gesamtkonzept zusammengeführt. Eine zentrale Rolle spielt dabei der Vergleich möglicher Hardwarekomponenten unter Berücksichtigung definierter Bewertungskriterien, welcher durch eine Nutzwertanalyse unterstützt wird.

Ein wesentlicher Bestandteil des Konzepts ist der Systementwurf. In diesem wird die Integration der verwendeten Technologien – wie z.\,B. NFC-Komponenten, die Anbindung der SPS über MQTT sowie die Einbettung in das Gesamtsystem – detailliert beschrieben. Dabei werden auch Aspekte wie die Strukturierung von Speicherbereichen auf NFC-Tags, die Auswahl geeigneter Softwarebibliotheken und die Kommunikationsarchitektur berücksichtigt. 

Ziel ist es, eine nachvollziehbare und belastbare Begründung für die gewählte Vorgehensweise zu liefern, die die Anforderungen der Anwendung erfüllt und eine verlässliche Grundlage für die anschließende Umsetzung bietet.


\section{Anforderungsanalyse mittels Requirements Engineering}
\label{sec:anforderungsanalyse}

\section{Hardwarebewertung mithilfe einer Nutzwertanalyse}
\label{sec:hardwarebewertung}

\section{Systementwurf}
\label{sec:systementwurf}